\documentclass{article}
\usepackage[utf8]{inputenc}
\usepackage[acronym]{glossaries}


\makeglossaries

\newglossaryentry{latex}
{
   name=latex,
   description={LaTeX (short for Lamport TeX) is a document preparation system. The user has to think about only the content to put in the document and the software will take care of the formatting. }
}

\newglossaryentry{glsy}
{
   name=glossary,
   description={Acronyms and terms which are generally unknown or new to common readers.}
}

\newacronym{s2e}{S2E}{Start to End}
\newacronym{b2e}{B2E}{Beginner to Expert}



\title{How to add glossary in LaTeX}
\author{ }
\date{ }

\begin{document}
\maketitle

This \Gls{latex} tutorial is very important and will talk about the use of \gls{glsy}. 
Plural of \Gls{glsy} is not \Glspl{glsy}. 

This tutorial which teach you glossary use from \acrlong{s2e}, which is  later abbreviated as \acrshort{s2e}  in this tutorial. This tutorial is useful for \acrlong{b2e}.


Yey!

\clearpage

\printglossary[type=\acronymtype]

\end{document}