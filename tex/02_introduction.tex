\section{Introduction}

{\huge{Subject: UC1ST1103 - Studio 1}}

\subsection{Document Information}

\begin{table}[h]
% \begin{tabular}{r @{ } c}
    \renewcommand{\arraystretch}{2}
    \begin{tabular}{r|l}
        Studio1 project deliverable: & Analysis of individual criteria \\
        Topic/Subject of analysis: & Critical Infrastructure Maturity\\
        Word count: & {\color{red}{N/A}}\\
        Estimated read time: & {\color{red}{HH:MM}} \\
        Pages: & \pageref{LastPage} \\
    \end{tabular}
    \renewcommand{\arraystretch}{1}
    
    \subsection{Course Lecturers}
    
    \renewcommand{\arraystretch}{2}
    % \begin{tabular}{r|lp{70mm}}
    \begin{tabular}{r|l}
        Course leader: & Prof. Mariya Chirchenkova Prof \\
        Course lecturer: & Prof. Rayne Reid \\
    \end{tabular}
    \renewcommand{\arraystretch}{1}
        
\end{table}

\subsection{Subject Matter}

This paper is an analysis of Norways Cyber Security posture. We will delve into; What is Cyber Security? How Cyber Security affects a nations, any nation and its citizens? Examine the challenges of most imminent import. And assess what lies further in the foreseable future? From the perpective Norways national Cyber Securities interest?

The latter part of the paper is a comparrable analysis of Norways National Cyber Security Strategy with its peers within the European Union. The basis of comparrison will focus on a selection of strategic imperatives as outlined by ENISA \footnote{European Union Agency for Cybersecurity}, in their {\emph{}{"NCSS"}}\footnote{National Cyber Security Strategy} Good Practice Guideline.

One focus area is on the analysis involving "Critical Information Infrastruture"; the underlying systems which provides services that everyone relies on, on a daily basis. Heed the previous statement as indication to how broad and vast the topic is. Therefore the necessetity to constraint on 2 specific area or industries.

Cyber Security concerning:
\begin{itemize}
    \item The financial industry
    \item The telecommunication industry
\end{itemize}

The countries subject to analysis and comparison are:
\begin{itemize}
    \item Norway
    \item Sweden
    \item Denmark
    \item Finland
    \item Iceland
    \item United Kingdom
    \item Russia
    \item France
    \item The Netherlands
    \item Bulgaria
\end{itemize}

The countries are selected on the basis of:
\begin{itemize}
    \item Geographical vicinity
    \item Similarities and/or contrastring traits in:
    \begin{itemize}
        \item IT infrastructure maturity
        \item IT services utilization
        \item Culture
        \item Geopolitical profile
        \item (Assumed) Cyber Security posture
        \item Compliance level (according to ENISA's NCSS Good Practice)
    \end{itemize}
\end{itemize}

\subsection{Exclusions}

The exact definition of what entails "Critical Information Infrastruture" is a fundamental necessetity to have established. It is therefore throughly outlined in the guidelines and directves used as source materials this analysis is based on.

This paper follows the source materials' definitons of {\emph{}{"cyber security"}}, financial industry and telecommunication industry. Other industries and security concerns may be mentioned, but not be the focus of the analysis.

Other countries, than the 10 listed above, maybe mentioned or referenced, but are otherwise not the focus of the analysis and comparison.

\subsection{Circle of trust}

Security, either physical or on the internat, is ultimately about establishing, balancing and managing a "Circle of trust".

There is no 1 solution that can be implemented and enfoced to 100\% secure any physical or digital entity.

There will always be a trade-off in terms of accepted risk, exposure, convenience and accessibility. And limitations to address in terms of human skill, time, IT resources and funding. Just to name a few.

